%%%%%%%%%%%%%%%%%%%%%%%%%%%%%%%%%%%%%%%%%%%%%%%%%%%%%%%%%%%%%%%%%%
% Package
%%%%%%%%%%%%%%%%%%%%%%%%%%%%%%%%%%%%%%%%%%%%%%%%%%%%%%%%%%%%%%%%%%
\usepackage{bxdpx-beamer}% dvipdfmxなので必要
\usepackage{amsthm}
\usepackage{amsmath}
\usepackage{amssymb}
\usepackage{amsfonts}
\usepackage{latexsym}
\usepackage{mathrsfs}
\usepackage{mathtools} % \coloneqq,\mathtoolsset
\usepackage{mleftright}
\usepackage{graphicx, xcolor}
\usepackage{float} % for figure
% \usepackage{hyperref} % beamer automatically loads hyperref.

%%%%%%%%%%%%%%%%%%%%%%%%%%%%%%%%%%%%%%%%%%%%%%%%%%%%%%%%%%%%%%%%%%
% Fonts and Encoding
%%%%%%%%%%%%%%%%%%%%%%%%%%%%%%%%%%%%%%%%%%%%%%%%%%%%%%%%%%%%%%%%%%
\usepackage[T1]{fontenc}
\usepackage{lmodern}
% \usepackage{newtxtext}
\usepackage{bm} % must be loaded after font change package.
\usepackage[uplatex,deluxe]{otf}
\usepackage[noalphabet]{pxchfon}
\setlightminchofont{HaranoAjiMincho-Light.otf}
\setmediumminchofont{HaranoAjiMincho-Medium.otf}
\setboldminchofont{HaranoAjiMincho-Bold.otf}
\setmediumgothicfont{HaranoAjiGothic-Medium.otf}
\setboldgothicfont{HaranoAjiGothic-Bold.otf}
\setxboldgothicfont{HaranoAjiGothic-Heavy.otf}
\setmarugothicfont{HaranoAjiGothic-Regular.otf}

\renewcommand{\kanjifamilydefault}{\gtdefault}
\renewcommand{\familydefault}{\sfdefault}
\renewcommand{\seriesdefault}{\bfdefault}
\usefonttheme[onlymath]{serif}
\mathversion{bold}
% PDFのしおり機能の日本語文字化けを防ぐ
\usepackage{pxjahyper}




%%%%%%%%%%%%%%%%%%%%%%%%%%%%%%%%%%%%%%%%%%%%%%%%%%%%%%%%%%%%%%%%%%
% Style
%%%%%%%%%%%%%%%%%%%%%%%%%%%%%%%%%%%%%%%%%%%%%%%%%%%%%%%%%%%%%%%%%%
\everymath{\displaystyle}

% Equation number
\numberwithin{equation}{section}
\mathtoolsset{showonlyrefs=true}

\renewcommand{\baselinestretch}{1.2}

\usepackage{footnpag}
\renewcommand\thefootnote{\arabic{footnote})}

%%%%%%%%%%%%%%%%%%%%%%%%%%%%%%%%%%%%%%%%%%%%%%%%%%%%%%%%%%%%%%%%%
% Aliases
%%%%%%%%%%%%%%%%%%%%%%%%%%%%%%%%%%%%%%%%%%%%%%%%%%%%%%%%%%%%%%%%%
% Math operator
\renewcommand{\Re}{\operatorname{Re}}
\renewcommand{\Im}{\operatorname{Im}}
%%% limit type
\DeclareMathOperator*{\argmin}{arg~min}
\DeclareMathOperator*{\argmax}{arg~max}
%%% log type
\DeclareMathOperator{\Var}{Var}
\DeclareMathOperator{\Cov}{Cov}
\DeclareMathOperator{\sgn}{sgn}
\DeclareMathOperator{\Supp}{Supp}
\DeclareMathOperator{\tr}{tr}
\DeclareMathOperator{\Tr}{Tr}
\DeclareMathOperator{\Det}{Det}
\DeclareMathOperator{\Log}{Log}
\DeclareMathOperator{\rank}{rank}
\DeclareMathOperator{\rk}{rk}
\DeclareMathOperator{\corank}{corank}
\DeclareMathOperator{\Res}{Res}
\DeclareMathOperator{\Ker}{Ker}
\DeclareMathOperator{\coker}{coker}
\DeclareMathOperator{\Coker}{Coker}
\DeclareMathOperator{\id}{id}
\newcommand{\eps}{\ensuremath{\varepsilon}}
\newcommand{\up}{\ensuremath{\uparrow}}
\newcommand{\down}{\ensuremath{\downarrow}}
\newcommand{\setmid}[2]{\left\{ #1 \mathrel{} \middle| \mathrel{} #2 \right\}} % set

% for probability
\newcommand{\indep}{\mathop{\rotatebox[origin=a]{90}{$ \models $}}} % independent
\newcommand{\laweqq}{\stackrel{\mathrm{law}}{=}}
\newcommand{\lawto}{\stackrel{\mathrm{law}}{\longrightarrow}}
\newcommand{\deqq}{\stackrel{\mathrm{d}}{=}}
\newcommand{\dto}{\stackrel{\mathrm{d}}{\longrightarrow}}
\newcommand{\wto}{\stackrel{\mathrm{w}}{\longrightarrow}}
\newcommand{\fdto}{\stackrel{\mathrm{f.d.}}{\longrightarrow}}
\newcommand{\parm}[2]{{}_{#1}\mathrm{P}_{#2}}
\newcommand{\comb}[2]{{}_{#1}\mathrm{C}_{#2}}
\newcommand{\hcomb}[2]{{}_{#1}\mathrm{H}_{#2}}

% Bracket
% required mathtools
\DeclarePairedDelimiter{\abs}{\lvert}{\rvert} % | | absolute value
\DeclarePairedDelimiter{\norm}{\lVert}{\rVert} % || || norm
\DeclarePairedDelimiter{\rbra}{(}{)} % () round brackets or parentheses
\DeclarePairedDelimiter{\cbra}{\{}{\}} % {} curly brackets or braces
\DeclarePairedDelimiter{\sbra}{[}{]} % [] square brackets or brackets
\DeclarePairedDelimiter{\abra}{\langle}{\rangle} % < > angle brackets or chevrons
\DeclarePairedDelimiter{\floor}{\lfloor}{\rfloor} % floor function
\DeclarePairedDelimiter{\ceil}{\lceil}{\rceil} % ceil function


% Alphabet
\newcommand{\bA}{\ensuremath{\mathbb{A}}}
\newcommand{\bB}{\ensuremath{\mathbb{B}}}
\newcommand{\bC}{\ensuremath{\mathbb{C}}}
\newcommand{\bD}{\ensuremath{\mathbb{D}}}
\newcommand{\bE}{\ensuremath{\mathbb{E}}}
\newcommand{\bF}{\ensuremath{\mathbb{F}}}
\newcommand{\bG}{\ensuremath{\mathbb{G}}}
\newcommand{\bH}{\ensuremath{\mathbb{H}}}
\newcommand{\bI}{\ensuremath{\mathbb{I}}}
\newcommand{\bJ}{\ensuremath{\mathbb{J}}}
\newcommand{\bK}{\ensuremath{\mathbb{K}}}
\newcommand{\bL}{\ensuremath{\mathbb{L}}}
\newcommand{\bM}{\ensuremath{\mathbb{M}}}
\newcommand{\bN}{\ensuremath{\mathbb{N}}}
\newcommand{\bO}{\ensuremath{\mathbb{O}}}
\newcommand{\bP}{\ensuremath{\mathbb{P}}}
\newcommand{\bQ}{\ensuremath{\mathbb{Q}}}
\newcommand{\bR}{\ensuremath{\mathbb{R}}}
\newcommand{\bS}{\ensuremath{\mathbb{S}}}
\newcommand{\bT}{\ensuremath{\mathbb{T}}}
\newcommand{\bU}{\ensuremath{\mathbb{U}}}
\newcommand{\bV}{\ensuremath{\mathbb{V}}}
\newcommand{\bW}{\ensuremath{\mathbb{W}}}
\newcommand{\bX}{\ensuremath{\mathbb{X}}}
\newcommand{\bY}{\ensuremath{\mathbb{Y}}}
\newcommand{\bZ}{\ensuremath{\mathbb{Z}}}

\newcommand{\bbA}{\ensuremath{\mathbf{A}}}
\newcommand{\bbB}{\ensuremath{\mathbf{B}}}
\newcommand{\bbC}{\ensuremath{\mathbf{C}}}
\newcommand{\bbD}{\ensuremath{\mathbf{D}}}
\newcommand{\bbE}{\ensuremath{\mathbf{E}}}
\newcommand{\bbF}{\ensuremath{\mathbf{F}}}
\newcommand{\bbG}{\ensuremath{\mathbf{G}}}
\newcommand{\bbH}{\ensuremath{\mathbf{H}}}
\newcommand{\bbI}{\ensuremath{\mathbf{I}}}
\newcommand{\bbJ}{\ensuremath{\mathbf{J}}}
\newcommand{\bbK}{\ensuremath{\mathbf{K}}}
\newcommand{\bbL}{\ensuremath{\mathbf{L}}}
\newcommand{\bbM}{\ensuremath{\mathbf{M}}}
\newcommand{\bbN}{\ensuremath{\mathbf{N}}}
\newcommand{\bbO}{\ensuremath{\mathbf{O}}}
\newcommand{\bbP}{\ensuremath{\mathbf{P}}}
\newcommand{\bbQ}{\ensuremath{\mathbf{Q}}}
\newcommand{\bbR}{\ensuremath{\mathbf{R}}}
\newcommand{\bbS}{\ensuremath{\mathbf{S}}}
\newcommand{\bbT}{\ensuremath{\mathbf{T}}}
\newcommand{\bbU}{\ensuremath{\mathbf{U}}}
\newcommand{\bbV}{\ensuremath{\mathbf{V}}}
\newcommand{\bbW}{\ensuremath{\mathbf{W}}}
\newcommand{\bbX}{\ensuremath{\mathbf{X}}}
\newcommand{\bbY}{\ensuremath{\mathbf{Y}}}
\newcommand{\bbZ}{\ensuremath{\mathbf{Z}}}

\newcommand{\cA}{\ensuremath{\mathcal{A}}}
\newcommand{\cB}{\ensuremath{\mathcal{B}}}
\newcommand{\cC}{\ensuremath{\mathcal{C}}}
\newcommand{\cD}{\ensuremath{\mathcal{D}}}
\newcommand{\cE}{\ensuremath{\mathcal{E}}}
\newcommand{\cF}{\ensuremath{\mathcal{F}}}
\newcommand{\cG}{\ensuremath{\mathcal{G}}}
\newcommand{\cH}{\ensuremath{\mathcal{H}}}
\newcommand{\cI}{\ensuremath{\mathcal{I}}}
\newcommand{\cJ}{\ensuremath{\mathcal{J}}}
\newcommand{\cK}{\ensuremath{\mathcal{K}}}
\newcommand{\cL}{\ensuremath{\mathcal{L}}}
\newcommand{\cM}{\ensuremath{\mathcal{M}}}
\newcommand{\cN}{\ensuremath{\mathcal{N}}}
\newcommand{\cO}{\ensuremath{\mathcal{O}}}
\newcommand{\cP}{\ensuremath{\mathcal{P}}}
\newcommand{\cQ}{\ensuremath{\mathcal{Q}}}
\newcommand{\cR}{\ensuremath{\mathcal{R}}}
\newcommand{\cS}{\ensuremath{\mathcal{S}}}
\newcommand{\cT}{\ensuremath{\mathcal{T}}}
\newcommand{\cU}{\ensuremath{\mathcal{U}}}
\newcommand{\cV}{\ensuremath{\mathcal{V}}}
\newcommand{\cW}{\ensuremath{\mathcal{W}}}
\newcommand{\cX}{\ensuremath{\mathcal{X}}}
\newcommand{\cY}{\ensuremath{\mathcal{Y}}}
\newcommand{\cZ}{\ensuremath{\mathcal{Z}}}

\newcommand{\sA}{\ensuremath{\mathscr{A}}}
\newcommand{\sB}{\ensuremath{\mathscr{B}}}
\newcommand{\sC}{\ensuremath{\mathscr{C}}}
\newcommand{\sD}{\ensuremath{\mathscr{D}}}
\newcommand{\sE}{\ensuremath{\mathscr{E}}}
\newcommand{\sF}{\ensuremath{\mathscr{F}}}
\newcommand{\sG}{\ensuremath{\mathscr{G}}}
\newcommand{\sH}{\ensuremath{\mathscr{H}}}
\newcommand{\sI}{\ensuremath{\mathscr{I}}}
\newcommand{\sJ}{\ensuremath{\mathscr{J}}}
\newcommand{\sK}{\ensuremath{\mathscr{K}}}
\newcommand{\sL}{\ensuremath{\mathscr{L}}}
\newcommand{\sM}{\ensuremath{\mathscr{M}}}
\newcommand{\sN}{\ensuremath{\mathscr{N}}}
\newcommand{\sO}{\ensuremath{\mathscr{O}}}
\newcommand{\sP}{\ensuremath{\mathscr{P}}}
\newcommand{\sQ}{\ensuremath{\mathscr{Q}}}
\newcommand{\sR}{\ensuremath{\mathscr{R}}}
\newcommand{\sS}{\ensuremath{\mathscr{S}}}
\newcommand{\sT}{\ensuremath{\mathscr{T}}}
\newcommand{\sU}{\ensuremath{\mathscr{U}}}
\newcommand{\sV}{\ensuremath{\mathscr{V}}}
\newcommand{\sW}{\ensuremath{\mathscr{W}}}
\newcommand{\sX}{\ensuremath{\mathscr{X}}}
\newcommand{\sY}{\ensuremath{\mathscr{Y}}}
\newcommand{\sZ}{\ensuremath{\mathscr{Z}}}


% %%%%%%%%%%%%%%%%%%%%%%%%%%%%%%%%%%%%%%%%%%%%%%%%%%%%%%%%%%%%%%%%%%
% % Theorems by amsthm package
% % See https://ftp.yz.yamagata-u.ac.jp/pub/CTAN/macros/latex/required/amscls/doc/amsthdoc.pdf.
% %%%%%%%%%%%%%%%%%%%%%%%%%%%%%%%%%%%%%%%%%%%%%%%%%%%%%%%%%%%%%%%%%%
% \newtheoremstyle{mystyle}%         % Name
%   {18pt}%                                     % Space above
%   {18pt}%                                     % Space below
%   {\normalfont}%                          % Body font
%   {}%                                     % Indent amount
%   {\bfseries}%                            % Theorem head font
%   {.}%                                    % Punctuation after theorem head
%   {6pt}%                                    % Space after theorem head, ' ', or \newline
%   {\thmname{#1}\thmnumber{ #2}\thmnote{ (#3)}}%
% \theoremstyle{mystyle}

\setbeamertemplate{theorems}[numbered]  %% 定理に番号をつける
\theoremstyle{definition}
\newtheorem{Thm}{定理}
\newtheorem{Lem}[Thm]{補題}
\newtheorem{Prop}[Thm]{命題}
\newtheorem{Cor}[Thm]{系}
\newtheorem{Def}[Thm]{定義}
\theoremstyle{example}
\newtheorem{Rem}[Thm]{注}
\newtheorem{Conj}[Thm]{予想}
\newtheorem{Ass}[Thm]{仮定}
\newtheorem{Eg}[Thm]{例}
\newtheorem{Egprob}[Thm]{例題}
\newtheorem{Prac}{練習}
\newtheorem{Prob}{問}
\newtheorem{Ex}{演習}
\newtheorem{Spprob}{特別課題}

\newtheorem*{Thm*}{定理}
\newtheorem*{Lem*}{補題}
\newtheorem*{Prop*}{命題}
\newtheorem*{Cor*}{系}
\newtheorem*{Ass*}{仮定}
\newtheorem*{Conj*}{予想}
\newtheorem*{Def*}{定義}
\newtheorem*{Rem*}{注}
\newtheorem*{Eg*}{例}
\newtheorem*{Egprob*}{例題}
\newtheorem*{Prac*}{練習}
\newtheorem*{Prob*}{問}
\newtheorem*{Ex*}{演習}
\newtheorem*{Spprob*}{特別課題}

\renewcommand{\proofname}{証明}

% 証明の部分をイタリックにせず,太字にする.
% 日本語しか書かないなら
% \renewcommand{\proofname}{\textbf{証明}}
% でよいが,それだと \begin{proof}[hoge] でhogeに数字やアルファベットなどを含むときはうまくいかない.
\makeatletter
\renewenvironment{proof}[1][\proofname]{\par
\pushQED{\qed}%
\normalfont \topsep6\p@\@plus6\p@\relax
\trivlist
\item\relax
% {\itshape
{\bfseries
#1\@addpunct{.}}\hspace\labelsep\ignorespaces
}{%
\popQED\endtrivlist\@endpefalse
}
\makeatother
