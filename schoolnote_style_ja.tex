%%%%%%%%%%%%%%%%%%%%%%%%%%%%%%%%%%%%%%%%%%%%%%%%%%%%%%%%%%%%%%%%%%
% Fonts and Encoding
%%%%%%%%%%%%%%%%%%%%%%%%%%%%%%%%%%%%%%%%%%%%%%%%%%%%%%%%%%%%%%%%%%
\usepackage[uplatex,deluxe]{otf}
\usepackage[noalphabet]{pxchfon}
\setlightminchofont{HaranoAjiMincho-Light.otf} % \mcfamily\ltseries
\setmediumminchofont{HaranoAjiMincho-Regular.otf} % \mcfamily\mdseries
\setboldminchofont{HaranoAjiMincho-Bold.otf} % \mcfamily\bfseries
\setmediumgothicfont{HaranoAjiGothic-Medium.otf} % \gtfamily\mdseries
\setboldgothicfont{HaranoAjiGothic-Medium.otf} % \gtfamily\bfseries
\setxboldgothicfont{HaranoAjiGothic-Heavy.otf} % \gtfamily\ebseries
\setmarugothicfont{HaranoAjiGothic-Regular.otf} % \mgfamily

\renewcommand{\textbf}[1]{{\bfseries\gtfamily#1}}
\usepackage{titlesec}
\titleformat*{\section}{\Large\bfseries\gtfamily}
\titleformat*{\subsection}{\large\bfseries\gtfamily}
\titleformat*{\subsubsection}{\bfseries\gtfamily}



%%%%%%%%%%%%%%%%%%%%%%%%%%%%%%%%%%%%%%%%%%%%%%%%%%%%%%%%%%%%%%%%%%
% Theorems by amsthm package
% See https://ftp.yz.yamagata-u.ac.jp/pub/CTAN/macros/latex/required/amscls/doc/amsthdoc.pdf.
%%%%%%%%%%%%%%%%%%%%%%%%%%%%%%%%%%%%%%%%%%%%%%%%%%%%%%%%%%%%%%%%%%
\newtheoremstyle{mystyle}%         % Name
  {18pt}%                                     % Space above
  {18pt}%                                     % Space below
  {\normalfont}%                          % Body font
  {}%                                     % Indent amount
  {\bfseries\gtfamily}%                            % Theorem head font
  {.}%                                    % Punctuation after theorem head
  {6pt}%                                    % Space after theorem head, ' ', or \newline
  {\thmname{#1}\thmnumber{ #2}\thmnote{ (#3)}}%
\theoremstyle{mystyle}

\newtheorem{Thm}{定理}[section]
\newtheorem{Lem}[Thm]{補題}
\newtheorem{Prop}[Thm]{命題}
\newtheorem{Cor}[Thm]{系}
\newtheorem{Ass}[Thm]{仮定}
\newtheorem{Conj}[Thm]{予想}
\newtheorem{Def}[Thm]{定義}
\newtheorem{Rem}[Thm]{注}
\newtheorem{Eg}[Thm]{例}
\newtheorem{Egprob}[Thm]{例題}
\newtheorem{Prac}{練習}[section]
\newtheorem{Prob}{問}
\newtheorem{Ex}{演習}[section]
\newtheorem{Spprob}{特別課題}

\newtheorem*{Thm*}{定理}
\newtheorem*{Lem*}{補題}
\newtheorem*{Prop*}{命題}
\newtheorem*{Cor*}{系}
\newtheorem*{Ass*}{仮定}
\newtheorem*{Conj*}{予想}
\newtheorem*{Def*}{定義}
\newtheorem*{Rem*}{注}
\newtheorem*{Eg*}{例}
\newtheorem*{Egprob*}{例題}
\newtheorem*{Prac*}{練習}
\newtheorem*{Prob*}{問}
\newtheorem*{Ex*}{演習}
\newtheorem*{Spprob*}{特別課題}

\renewcommand{\proofname}{証明}
\newenvironment{solution}{\begin{proof}[解]}{\end{proof}}

% 証明の部分をイタリックにせず,太字にする.
% 日本語しか書かないなら
% \renewcommand{\proofname}{\textbf{証明}}
% でよいが,それだと \begin{proof}[hoge] でhogeに数字やアルファベットなどを含むときはうまくいかない.
\makeatletter
\renewenvironment{proof}[1][\proofname]{\par
\pushQED{\qed}%
\normalfont \topsep6\p@\@plus6\p@\relax
\trivlist
\item\relax
% {\itshape
{\bfseries\gtfamily
#1\@addpunct{.}}\hspace\labelsep\ignorespaces
}{%
\popQED\endtrivlist\@endpefalse
}
\makeatother
